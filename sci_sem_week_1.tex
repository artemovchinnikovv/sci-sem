\documentclass{article}

\usepackage{color}   %May be necessary if you want to color links
\usepackage{hyperref}
\hypersetup{
    colorlinks=true, %set true if you want colored links
    linktoc=all,     %set to all if you want both sections and subsections linked
    linkcolor=blue,  %choose some color if you want links to stand out
}

\usepackage{amsmath}
\DeclareMathOperator{\Tr}{Tr}

\title{Initial overview of the L-unification}
\author{Artem Ovchinnikov}
\date{\today}

\begin{document}

\maketitle

\tableofcontents

\newpage
\section{Disadvantages of the standard model}
Non-sufficient self-consistency of the underlying version of quantum field theory. \\
Incapability of describing the gravitational sector at quantum level. \\
The standard model is too complicated to be the dreamlike distinguished fundamental theory. \\

\section{Two basic ways to create a new theory}
One can think that the particles of the standard model are not really elementary, but are made from a smaller number of simpler components – this leads to development of composite, e.g. technicolor models. \\
One can think that the particles of the standard model are only fragments of some bigger, more symmetric and universal entities – this is the idea behind unification models. \\

\section{Unification models}
The basic idea of all unification models is to enhance the symmetry, embedding the non-simple gauge group $ G_{st} = SU (3) \times SU (2) \times U (1) $ of the standard model into a larger simple group $G$, which inevitably has larger dimension and thus introduces extra particles.

\section{Main idea}
The central idea of Lisi’s approach is to consider a minimal modification – ideally just a reformulation – of the standard model, with as few additional fields as only possible. \\
No extra essences are introduced, like superpartners, extra dimensions, branes, strings etc. \\
Since the number of fields in the standard model is of the order of 200, the number of different generators of ”unification algebra” should exceed this number. If Lie algebra is classical, like $SU (N)$, $SO(N)$ or $Sp(N)$, this requires it to have a large rank $= \sqrt {200} > 14$. If instead the group is exceptional, then the smallest group with dimension $> 200$ is actually the largest: $E_8$ with $dim(E_8) = 248$ and its rank is as low as $rank(E_8) = 8$ (dimension of the next smaller exceptional group $dim(E7) = 133 < 200$).

\newpage
\section{L and G - unification}
Furthermore, while in G-unification models, in order to minimize the number of 3 additional gauge bosons, one attempts to find a group of the lowest dimension with a given rank, in L-unification for the same purpose one tries to minimize the rank for a given dimension, which is now defined by the number of standard-model particles, predominantly fermions. \\

\section{The number of known elementary particles}
Standard model fermions numbers $n_F = 96$ (degrees of freedom) and $N_F = 192$ (fields). \\
Standard model bosons numbers $n_B = 30$ and $N_B = 92$. \\

\section{Without extra particles and fields?}
Extra particles, are, however, necessary in any case: because of the rank mismatch: there is simply no Lie group which could both match the strict lower limit $ dim(G) \geq n_B + n_F > n_F = 90$ and saturate the rank condition $rank(G) = rank(U (1)) + rank(SU (2)) + rank(SU (3)) + rank(SO(3, 1)) = 1 + 1 + 2 + 2 = 6$.

\section{The main claim}
The action of the standard model coupled to Einstein gravity,
\begin{multline}
    \int \sqrt{\text{det}\, g} \, d^4x \{ M_{\text{Pl}}^2 R(g) - \Lambda + \frac{1}{4g_3^2} \text{Tr}_{SU(3)} G_{\mu\nu} G^{\mu\nu} + \frac{1}{4g_2^2} \text{Tr}_{SU(2)} W_{\mu\nu} W^{\mu\nu} + \frac{1}{4g_1^2} V_{\mu\nu} V^{\mu\nu} + \\ + \sum_{p=1}^{n_g=3} \left( \bar{l}^{(p)}_L \hat{D} l^{(p)}_L + \bar{l}^{(p)}_R \hat{D} l^{(p)}_R + \bar{q}^{(p)}_L \hat{D} q^{(p)}_L + \bar{q}^{(p)}_R \hat{D} q^{(p)}_R \right) + \frac{1}{2} D_\mu \phi^+ D^\mu \phi + V(\phi) + 
    \\ + \sum_{P,Q=1}^{2n_g=6} \left( M^{(PQ)}_l \bar{l}^{(P)}_R \phi l^{(Q)}_L + M^{(PQ)}_q \bar{q}^{(P)}_R \phi q^{(Q)}_L + \text{c.c.} \right) \}
\end{multline}
can be rewritten in a compact form as
\begin{equation}
    \int \text{Tr}_{E_8} B \wedge F + \int Q(B)
\end{equation}
where
\begin{equation}
    F = dA + A \wedge A
\end{equation}
is a curvature of the "generalized-connection" field $A(x)$, which is a linear combination of the standard-model fields, distributed over elements of the $E_8$-algebra matrix (adjoint representation of $E_8$), and $Q(B)$ is a quadratic function of the auxiliary field $B$, which is also an adjoint $E_8$ matrix.

In a little more detail:
\begin{equation}
    A = \sum_{\alpha \in G}^{56} A^\alpha(x) T_\alpha \oplus \sum_{a \in E/G}^{192} \psi^a(x) T_a,
\end{equation}
\begin{equation}
    B = \sum_{\alpha \in G}^{56} B^\alpha(x) T_\alpha \oplus \sum_{a \in E/G}^{192} \chi^a(x) T_a
\end{equation}
and
\begin{equation}
    Q(B) = \sum_{\alpha, \alpha' \in G}^{56} \left( Q_{\text{grav}}^{\alpha, \alpha'} B^\alpha \wedge B^{\alpha'} + Q_{\text{YM}}^{\alpha, \alpha'} B^\alpha \wedge *B^{\alpha'} \right)
\end{equation}

Here the sums are over 248 generators of $E_8$, which are divided into two different groups of 56 and 192, labeled by Greek and small Latin letters respectively, which correspond to decomposition of adjoint (the minimal possible) representation of $E_8$ into a subalgebra $G = SO(7,1) \times SO(8)$ and its representation $R = E/G$.\\ Bosonic 1-forms $A^\alpha = A^\alpha_\mu dx^\mu$ include the gauge fields of the standard model, the gravitational spin-connection and a mixture of Higgs scalars and gravitational tetrades. One of the two tetrade indices is included into the set of $E_8$-indices $\{\alpha\}$, the remaining vector index, as well as the one of the gauge vectors is the 1-form index of $A$. \\
Fermionic (Grassmann-valued) 0-forms $\psi^a$ and $\chi^a$ are made from the fermions of the standard model, with spinor indices included into the set of $\{a\}$. Auxiliary fields $B^\alpha = B^\alpha_{\mu\nu} dx^\mu \wedge dx^\nu$ and $\chi^a = \chi^a_{\mu\nu\lambda} dx^\mu \wedge dx^\nu \wedge dx^\lambda$ are bosonic 2-forms and fermionic 3-forms respectively. No 4-form component is included into $B$, therefore the $\psi^2$ term in $F = F^\alpha T_\alpha + (D\psi)^a T_a + \psi^a \psi^b [T_a, T_b]$ does not contribute to the action. \\
The main disadvantage is, however, more serious: $E_8$ unification-model not fully justified and, most probably, is simply wrong. Desired distribution of all the fields of the standard model among the generators of $E8$ is not actually found. \\

\newpage
\section{Main drawbacks}
Introduction of new fields beyond the standard model is still unavoidable. \\
Hierarchy and of quantum gravity are not resolved. \\
Cosmological constant is non-vanishing in BF -version of the Palatini formalism. \\
G-projector and the Hodge star, appear in the action -- not fully topological. \\
Higher generations are not adequately described. \\

\section{Features of L-unification}
Unification algebra $G = SO(7, 1) \times SO(8)$ contains gravitational $SO(3, 1) = SL(2, C)$ and Yang-Mills $G_{YM} = SU (3) × (SU (2) \times U (1)) \times 2$. \\
This is a vector-like theory, with right-hand copies of $W±$, $Z$ bosons and second photon – which should somehow decouple at low energies: along with other gauge bosons from $G/G_{st}$ should acquire large masses. \\
Unification group $G = SO(7, 1) \times SO(8)$ is not simple, with strong and electroweak groups belonging to different factors – $SO(8)$ and $SO(7, 1)$ respectively, – there is no danger of proton decay. \\
The same non-simplicity implies that unification of coupling constants $g_{str}$ and $g_{weak}$ does not need to take place in this model. \\
Flavor-changing neutral currents. \\

\section{From Palatini action to Einstein equations}
In the case of arbitrary space-time dimension d Palatini action involves the curvature of SO(d) connection
\begin{equation}
    S_p = \epsilon_{a_1 ... a_d} \int \exp{a_1} \land ... \land \exp{a_{d-2}} \land R^{a_{d-1}a_d}
\end{equation}
\begin{equation}
    ...
\end{equation}
\begin{equation}
    R_\mu^a + \frac{1}{2}e_\mu^a R \sim \det(e)T_\mu^a - e^a_\mu T
\end{equation}
It is easy to deduce that $F = -T$ and therefore previous equation is actually equivalent to the proper Einstein equation

\newpage
\section{Decompositions of $E_8$ algebra}
\begin{equation}
    E_8 = SO(16) + R_{128}
\end{equation}
where $R_{128}$ is spinor representation of $SO(16)$.
\begin{equation}
    E_8 = D_8 + R_{128} = D_4 + D_4 + (8 \times 8 + R_{128})
\end{equation}
Gauge fields are distributed among the two $D_4$ factors, with gravity and two weak groups (left and right) belonging to one of them and with strong and two abelian groups belonging to another. \\
Fermionic fields are distributed among the spinor representation $R_128$ of maximal subgroup $SO(16) \subset E8$ and the $8 \times 8$ generators of the factor $D_8/D_4 \times D_4$. The three 64-plet constituents of the fermionic sector are related by discrete triality symmetry. \\
\end{document}

The rank of the standard-model gauge group, even after gravitational sector is included, $G_{st+gr} = SU (3) \times SU (2) \times U (1) \times SO(3, 1)$ is still 6, i.e. less than the rank 8 of the minimal possible ”unification group” E8. The deficit rank 2 is placed into additional gauge group –- the right-hand copy of the Weinberg-Salam group, giving rise to the right-hand copies.
