\documentclass{article}

\usepackage{color}   %May be necessary if you want to color links
\usepackage{hyperref}
\hypersetup{
    colorlinks=true, %set true if you want colored links
    linktoc=all,     %set to all if you want both sections and subsections linked
    linkcolor=blue,  %choose some color if you want links to stand out
}

\usepackage{amsmath}
\DeclareMathOperator{\Tr}{Tr}

\title{Initial overview of the L-unification}
\author{Artem Ovchinnikov}
\date{\today}

\begin{document}

\maketitle

\section{Introduction}
Non-sufficient self-consistency of the underlying version of quantum field theory. \\
Incapability of describing the gravitational sector at quantum level. \\
The standard model is too complicated. $\xrightarrow{}$ \\
One can think that the particles of the standard model are made from a smaller number of simpler components – technicolor models. \\
The particles of the standard model are fragments of some bigger, more symmetric and universal entities – unification models. $\xrightarrow{}$ \\
The basic idea of all unification models is to enhance the symmetry, embedding the non-simple gauge group $ G_{st} = SU (3) \times SU (2) \times U (1) $ of the standard model into a larger simple group $G$.

\section{Central idea}
The central idea of Lisi’s approach is to consider a StM minimal modification. \\
Number of fields in the standard model is of the order of 200 -- the number of different generators. \\
If Lie algebra is classical -- large rank $= \sqrt {200} > 14$. \\
If Lie group is exceptional -- $E_8$ (the next smaller exceptional group $dim(E7) = 133 < 200$). \\

\section{Without extra particles and fields?}
StM fermions numbers $n_F = 96$ (degrees of freedom) and $N_F = 192$ (fields). \\
StM bosons numbers $n_B = 30$ and $N_B = 92$. \\
Extra particles necessary anyaway: there is simply no Lie group which could both match dimensions and the rank condition. ???

\section{The main claim}
The action of the standard model coupled to Einstein gravity, ???
\begin{multline}
    \int \sqrt{\text{det}\, g} \, d^4x \{ M_{\text{Pl}}^2 R(g) - \Lambda + \frac{1}{4g_3^2} \text{Tr}_{SU(3)} G_{\mu\nu} G^{\mu\nu} + \frac{1}{4g_2^2} \text{Tr}_{SU(2)} W_{\mu\nu} W^{\mu\nu} + \frac{1}{4g_1^2} V_{\mu\nu} V^{\mu\nu} + \\ + \sum_{p=1}^{n_g=3} \left( \bar{l}^{(p)}_L \hat{D} l^{(p)}_L + \bar{l}^{(p)}_R \hat{D} l^{(p)}_R + \bar{q}^{(p)}_L \hat{D} q^{(p)}_L + \bar{q}^{(p)}_R \hat{D} q^{(p)}_R \right) + \frac{1}{2} D_\mu \phi^+ D^\mu \phi + V(\phi) + 
    \\ + \sum_{P,Q=1}^{2n_g=6} \left( M^{(PQ)}_l \bar{l}^{(P)}_R \phi l^{(Q)}_L + M^{(PQ)}_q \bar{q}^{(P)}_R \phi q^{(Q)}_L + \text{c.c.} \right) \}
\end{multline}
can be rewritten in a compact form as ???
\begin{equation}
    \int \text{Tr}_{E_8} B \wedge F + \int Q(B)
\end{equation}
\begin{equation}
    F = dA + A \wedge A
\end{equation}
field $A(x)$, which is a linear combination of the standard-model fields, distributed over elements of the $E_8$-algebra matrix (adjoint representation of $E_8$). \\
$Q(B)$ is a quadratic function of the auxiliary field $B$ (also an adjoint $E_8$ matrix). \\

\begin{equation}
    A = \sum_{\alpha \in G}^{56} A^\alpha(x) T_\alpha \oplus \sum_{a \in E/G}^{192} \psi^a(x) T_a,
\end{equation}
\begin{equation}
    B = \sum_{\alpha \in G}^{56} B^\alpha(x) T_\alpha \oplus \sum_{a \in E/G}^{192} \chi^a(x) T_a
\end{equation}
\begin{equation}
    Q(B) = \sum_{\alpha, \alpha' \in G}^{56} \left( Q_{\text{grav}}^{\alpha, \alpha'} B^\alpha \wedge B^{\alpha'} + Q_{\text{YM}}^{\alpha, \alpha'} B^\alpha \wedge *B^{\alpha'} \right)
\end{equation}

Here the sums are over 248 generators of $E_8$, which are divided into two different groups of 56 and 192, which correspond to decomposition of adjoint (the minimal possible) representation of $E_8$ into a subalgebra $G = SO(7,1) \times SO(8)$ and its representation $R = E/G$.\\ 
The main disadvantage is, however, more serious: desired distribution of all the fields of the standard model among the generators of $E_8$ is not actually found. ??? \\

\newpage
\section{Main drawbacks}
Introduction of new fields beyond the standard model is still unavoidable. \\
Hierarchy and of quantum gravity are not resolved. \\
Cosmological constant is non-vanishing in BF -version of the Palatini formalism. ???\\
G-projector and the Hodge star, appear in the action -- not fully topological. ???\\
Higher generations are not adequately described. \\

\section{Features of L-unification}
Right-hand copies of $W±$, $Z$ bosons and second photon – which should somehow decouple at low energies. ??? \\
Unification group $G = SO(7, 1) \times SO(8)$ is not simple, with strong and electroweak groups belonging to different factors -– $SO(8)$ and $SO(7, 1)$ respectively, – there is no danger of proton decay. ???\\

\section{From Palatini action to Einstein equations}
In the case of arbitrary space-time dimension d Palatini action involves the curvature of SO(d) connection
\begin{equation}
    S_p = \epsilon_{a_1 ... a_d} \int \exp{a_1} \land ... \land \exp{a_{d-2}} \land R^{a_{d-1}a_d}
\end{equation}
\begin{equation}
    ...
\end{equation}
\begin{equation}
    R_\mu^a + \frac{1}{2}e_\mu^a R \sim \det(e)T_\mu^a - e^a_\mu T
\end{equation}

\newpage
\section{Decompositions of $E_8$ algebra}
\begin{equation}
    E_8 = D_8 + R_{128} = D_4 + D_4 + (8 \times 8 + R_{128})
\end{equation}
where $R_{128}$ is spinor representation of $SO(16)$, $D_8 = SO(16)$. \\

Gauge fields are distributed among the two $D_4$ factors, with gravity and two weak groups (left and right) belonging to one of them and with strong and two abelian groups belonging to another. \\
Fermionic fields are distributed among the spinor representation $R_128$ of maximal subgroup $SO(16) \subset E8$ and the $8 \times 8$ generators of the factor $D_8/D_4 \times D_4$. The three 64-plet constituents of the fermionic sector are related by discrete triality symmetry. \\
\end{document}

The rank of the standard-model gauge group, even after gravitational sector is included, $G_{st+gr} = SU (3) \times SU (2) \times U (1) \times SO(3, 1)$ is still 6, i.e. less than the rank 8 of the minimal possible ”unification group” E8. The deficit rank 2 is placed into additional gauge group –- the right-hand copy of the Weinberg-Salam group, giving rise to the right-hand copies.

Unification algebra $G = SO(7, 1) \times SO(8)$ contains gravitational $SO(3, 1) = SL(2, C)$ and Yang-Mills $G_{YM} = SU (3) × (SU (2) \times U (1)) \times 2$. \\
